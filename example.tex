%% Poster template using tikzposter
%% see https://ctan.org/pkg/tikzposter  for a manual
%% and https://www.sharelatex.com/templates/53332341910d975953dffdab/v/1/pdf for an example
%% We (see MPIISstyle for authors) added some commands to this to make it even more convinient

%\documentclass[a0paper,portrait]{tikzposter}
\documentclass[a0paper,landscape]{tikzposter} % change paper layout/size
%% If you want a custom paper size use the following. Note that our printer has 106.7cm wide paperroll
%% If you have to print to the margin leave some space to cut it.
%% This would be a large landscape poster
%\documentclass[landscape]{tikzposter}
% \geometry{paperheight=140cm,paperwidth=104cm} % for landscape width and height are swapped

\usepackage{calc}
\usepackage{lipsum}
\usepackage[numbers, sort&compress, square]{natbib}
\usepackage{amsmath}
\usepackage{amssymb}
\usepackage{url}
\usepackage{enumitem}
 \usepackage{relsize}
% %if you need algorithm do this:
% \usepackage{algorithm}
% \usepackage{algorithmic}
% \usepackage{etoolbox}
% \AtBeginEnvironment{algorithm}{
%   \setlength{\columnwidth}{\linewidth}
% }

\usepackage[fontscale=1]{MPIISposter}
% in addition to the standard latex font sizes we have now also \veryHuge, \VeryHuge, and \VERYHuge
% there is a command
% \fontscale{factor} to scale all fonts

\usetitlestyle{Empty}
%% here you can change the way to boxed look like, see
%% https://bitbucket.org/surmann/tikzposter/downloads/styleguide.pdf
\useblockstyle{Envelope}

 %% add your color scheme in this file!
\definecolorstyle{MPI_rg}{
	\colorlet{colorPrimary}{rgb,255:red,75; green,168; blue,79}
	\colorlet{colorSecondary}{colorPrimary!50!black}
%	\colorlet{colorSecondary}{rgb,255:red,0; green,117; blue,103}
%	\colorlet{colorSecondary}{rgb,255:red,211; green,211; blue,204}
	\colorlet{colorFG}{black}
	\colorlet{colorBG}{white}
	\colorlet{colorFGheader}{white}
	\colorlet{colorBGheader}{black}
}{
	\colorlet{backgroundcolor}{colorBG}
	\colorlet{framecolor}{colorBG!50}
	\colorlet{titlefgcolor}{colorFGheader}
	\colorlet{titlebgcolor}{colorPrimary}
	\colorlet{blocktitlebgcolor}{colorPrimary}
	\colorlet{blocktitlefgcolor}{colorFGheader}
	\colorlet{blockbodybgcolor}{colorBG!10}
	\colorlet{blockbodyfgcolor}{colorFG}
	\colorlet{innerblocktitlebgcolor}{blocktitlebgcolor}
	\colorlet{innerblocktitlefgcolor}{blocktitlefgcolor}
	\colorlet{innerblockbodybgcolor}{blocktitlebgcolor!10}
	\colorlet{innerblockbodyfgcolor}{blockbodyfgcolor}
	\colorlet{notefgcolor}{colorFG}
	\colorlet{notebgcolor}{colorPrimary!20}
	\colorlet{notefrcolor}{colorPrimary}
}

\definecolorstyle{MPI_ei}{
	\colorlet{colorPrimary}{rgb,255:red,255; green,186; blue,77}
	\colorlet{colorSecondary}{rgb,255:red,0; green,117; blue,103}
	\colorlet{colorFG}{black}
	\colorlet{colorBG}{white}
	\colorlet{colorFGheader}{black}
	\colorlet{colorBGheader}{white}
}{
	\colorlet{backgroundcolor}{colorBG}
	\colorlet{framecolor}{colorBG!50}
	\colorlet{titlefgcolor}{colorFGheader}
	\colorlet{titlebgcolor}{colorPrimary}
	\colorlet{blocktitlebgcolor}{colorPrimary}
	\colorlet{blocktitlefgcolor}{colorFGheader}
	\colorlet{blockbodybgcolor}{colorBG!10}
	\colorlet{blockbodyfgcolor}{colorFG}
	\colorlet{innerblocktitlebgcolor}{blocktitlebgcolor}
	\colorlet{innerblocktitlefgcolor}{blocktitlefgcolor}
	\colorlet{innerblockbodybgcolor}{blocktitlebgcolor!10}
	\colorlet{innerblockbodyfgcolor}{blockbodyfgcolor}
	\colorlet{notefgcolor}{colorFG}
	\colorlet{notebgcolor}{colorPrimary!20}
	\colorlet{notefrcolor}{colorPrimary}
}
\usecolorstyle{MPI_rg} % load the respective color scheme
%\usecolorstyle{MPI_ei}

% there is a colorize command that make text appear in the primary color
% you might want to change it if your primary color is very light (here mixed with black 1:1)
%\renewcommand{\colorize}[1]{{\color{blocktitlebgcolor!50!black}\bf #1}}

\newcommand{\PosterDepartment}{Max Planck Institute for Intelligent Systems -- Autonomous Learning Group}
\newcommand{\PosterTitle}{ This is a Title}
\newcommand{\PosterAuthor}{Author1, Author2, and Author3}
\renewcommand{\PosterFooter}{al.is.tuebingen.mpg.de}


\renewcommand{\familydefault}{\sfdefault}
\title{\PosterTitle}
\author{\PosterAuthor}

\settitle{
  \s{1.5}
  \begin{center}
    \color{titlefgcolor}
    {\Huge\PosterDepartment\\[0.5em]} % adjust here depending on whether you have two rows somewhere
    {\VERYHuge \bfseries \PosterTitle\\[0.3em]}
    {\huge \PosterAuthor\\[.5em]}
  \end{center}
  \null
}

\begin{document}
\maketitle[titletotopverticalspace=0cm,titletoblockverticalspace=1.5cm]
%% decide about black or white Minerva
%% BTW: the Minerva looking into the poster is nicer than if she looks out, so on the left
\addlogoleft{logos/Max-Planck-Gesellschaft-white}
% \addlogoleft{logos/Max-Planck-Gesellschaft-black}

%% you can add another logo and shift it inwards if you want
%\addlogoright[5cm]{yourlogo}

\posterfooter


\begin{columns}
  \column{0.33}

  \specialblock{Discrete case}{
        \colorize{The model:}
        Let $G=(E,V)$ be graph. The Physarum Polycephalum adaptative model is
        \begin{myequations}
            \begin{align}
              & \underset{e\in \sigma(v)}{\sum}Q_e(t)=f_v,  \\
                &Q_e(t)=D_e(t)\dfrac{p_u(t)-p_v(t)}{L_e}  &\forall e\in E, \\
                &D'_e(t)=|Q_e(t)|^\beta-D_e(t) &\forall e\in E,
            \end{align}
        \end{myequations}
        
        \s{1.5}
       \begin{cols}
      \col{.5}
\innerblock{The parameters:}{
\s{.5}
\begin{enumerate}
    \item $f$ models the location of the food,
    \item $Q$ represents the flux through the slime mold edges,
    \item $D$ \textbf{is the capacity of the edges,}
    \item $p$ \textbf{is the pressure in the nodes,}
\end{enumerate}
\s{.5}
    }
      \s{.5}
      \innerblock{The problem:}{
\s{.5}

For each $t$, find a pair of functions  $D_e(t)$ and $p_v(t)$  that satisfies the PP equations.
\s{.5}
    }
    \s{.5}
\col{.4} 
\vspace{-.5}
\ig{pics/PP}
\s{.2}
\small{Tero, A., Kobayashi, R., & Nakagaki, T. (2007). \textbf{A mathematical model for adaptive transport network in path finding by true slime mold}. Journal of theoretical biology, 244(4), 553-564.}
    \end{cols} 
        
            \s{.3}
  }

  \block{Continuous case}{
    % \colorize{\bf Some nifty features:}
    % Here are some of the commands we added to make life easier:
    % \begin{enumerate} % verb// does not work inside of block
    % \item {\tt fontscale} argument allows to scale all fonts (easy to refit poster)
    % \item {\tt \textbackslash{}colorize\{test\}} makes text \colorize{colored} (automatically fitting to box color)
    % \item {\tt \textbackslash{}s\{X\}} create a vertical space in units of em (relative to fontsize)
    % \item {\tt cols} environment creating a multicolumn inside a box
    %   with {\tt \textbackslash{}col\{0.5\}} creating a new column with relative size
    % \item {\tt \textbackslash{}ig[relsize]\{img\}} short for includegraphics with relative width to column (default 1.0)
    % \end{enumerate}%

    % \innerblock{interesting}{

    %   Normal text or \colorize{colorized text}
    % }

    % \s{1} % add some spacing
    % \innerblock{}{
    %   box without header
    % }

    % \s{1}
    % placing pics works great with cols environment (like in beamer with columns)
    % \s{1}

\colorize{The model:}

The \textit{DMK} equations:
\begin{subequations}\label{eq:2}
    \begin{align}
&-\text{div}\left( \mu (t,x) \nabla u(t,x)\right)=f^+(x)-f^-(x)\\
         &\partial_t \mu(t,x)=\left(\mu(t,x)\nabla u(t,x)\right)^\beta-k(x)d(t)\mu^\alpha(t,x)
  \end{align}
\end{subequations}

complemented by zero Neumann boundary conditions. 


\s{1.5}
\begin{cols}
      \col{.4}
\innerblock{The parameters:}{
\s{.5}
\begin{enumerate}
    \item $f^+$ models the source
    \item $f^-$ corresponds to the sink
    \item  $k$ controls the barriers in the domain
    \item $d$ represents the decay 
    \item $\mu$ \textbf{is the transport density}
    \item $u$ \textbf{is called the potential }
\end{enumerate}
\s{.5}
    }
      
      \col{.4} 
      \innerblock{The problem:}{
\s{.5}

Find 
\vspace{-1}
$$\mu:[0,+\infty)\rightarrow  \mathbb{R}^+,$$
$$u :\Omega\rightarrow  \mathbb{R}^d,$$

satisfying the  \textit{DMK} equations.
\s{.5}
    }
    \s{.5}
% \innerblock{The problem:}{
% \s{.5}
% IMG
% \s{1.5}
%     }
     \end{cols}

 \s{1.}

}

  % COLUMN 2
  % ---------------------------------------------------------------------------
  \column{0.33}

  \block{Questions}{
  \begin{itemize}
      \item How the initial conditions of the DMK equations influence on the solution?
      \item How can we translate different solutions into a graph representation?
      \item Which are the topological properties of the optimal graphs?
  \end{itemize}
  \innerblock{Goals}{
\s{.5}
\begin{itemize}
    \item  To translate the solution pair $(\mu, u)$ into a network framework
    \item To characterize the topologies associated to this novel representation
\end{itemize}
}
}

  \specialblock{Current step:}{

\innerblock{}{A \textit{delta} initial density and different approximations of the solution:}
\s{.5}
    \begin{cols}
      \col{.32}
      \ig{pics/Poster_1.png}
      \col{.32} \centering
      \ig{pics/poster_2.png}
      \col{.32}
      \ig{pics/poster_3.png}

    \end{cols}
    
 \innerblock{}{The three graph representations of the solution:} 
    \begin{cols}
      \col{.3}
      \ig{pics/Graph_1_delta_1.png}
      \col{.3} \centering
     \ig{pics/Graph_2_delta_1.png}
      \col{.3}
       \ig{pics/Graph_3_delta_1.png}

    \end{cols}
    
    
\innerblock{}{A \textit{parabolic} initial density and different approximations of the solution:}
\s{.5}
    \begin{cols}
      \col{.32}
      \ig{pics/Poster_4.png}
      \col{.32} \centering
      \ig{pics/poster_5.png}
      \col{.32}
      \ig{pics/poster_6.png}

    \end{cols}
    

    
  }
  % COLUMN 3
  % ---------------------------------------------------------------------------
   \column{0.33}
   
   \specialblock{}{  
     \innerblock{}{The three graph representations of the solution:} 
    \begin{cols}
      \col{.3}
      \ig{pics/Graph_1_parabola.png}
      \col{.3} \centering
     \ig{pics/Graph_2_parabola.png}
      \col{.3}
       \ig{pics/Graph_3_parabola.png}

    \end{cols}    
    
   

}
 \specialblock{Future Steps}{
   \begin{itemize}
       \item Enhancing the graph representations by removing non-optimal structures
       \item Analysing such refined networks in terms of different metrics 
       \item Improving the DMK solver by exploiting these characterizations
   \end{itemize}
  }


  \block{}{
    % use bibtex here
    % \nocite*  % use this to force all entries in the bibfile to go here
    % \bibliography{mybib}
    % or the manual entries (can be copied from .bbl files in case you have used bibtex on a document)
    \setlength{\bibsep}{.5em}
    \begin{thebibliography}{1}

    \bibitem{Repo}
      Facca, E., Cardin, F., & Putti, M. (2018).
      \newblock Towards a Stationary Monge--Kantorovich Dynamics: The Physarum Polycephalum Experience.
      \newblock SIAM Journal on Applied Mathematics, 78(2), 651-676.

    \bibitem{tikzpackage}
      Richard Barnard, Elena Botoeva, Pascal Richter, and Dirk Surmann.
      \newblock tikzposter -- Create scientific posters using TikZ.
      \newblock \url{https://ctan.org/pkg/tikzposter}, 2014.

    \bibitem{usingtikz}
      Richard Barnard, Elena Botoeva, Pascal Richter, and Dirk Surmann.
      \newblock Using Tikzposter.
      \newblock \url{https://www.sharelatex.com/templates/53332341910d975953dffdab/v/1/pdf},

    \end{thebibliography}
  }

\end{columns}

\end{document}